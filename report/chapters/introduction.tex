\chapter{Introduction}
\label{cha:Introduction}

A program is a sequence of machine instructions which are written by a
human and executed by a computer. The VHDL language has been created to describe
digital and signals systems such as field-programmable gate arrays and
integrated circuits\cite{wiki-vhdl}.The problem is that VHDL is an old, verbose
and tricky language.

SpinalHDL has been created to offer a way to outmatch those disadvantages. The
main goal of a DSL (domain-specific language) is to offer a more pleasant way to
code and that's what does SpinalHDL. As written before, the code is written by a
human and by using SpinalHDL, the human becomes happier than using just the VHDL.

KlugHDL is another tool to make the human happier, it's a component diagram
generator which produce statics and dynamics diagrams.

\section{Context}
\label{sec:Context}

SpinalHDL is a programming language to describe digital hardware and generate
the corresponding source code in VHDL (or Verilog). SpinalHDL is written in Scala as a DSL
and has multiple advantages\cite{github-spinalhdl} :
\begin{itemize}
    \item No restriction to the genericity of hardware description by using
      Scala constructs
    \item No more endless wiring. Create and connect complex buses like AXI in one line.
    \item Evolving capabilities. Create your own buses definition and abstraction layer.
    \item Reduce code size by a high factor, especially for wiring. Allowing you
      to have a better visibility, more productivity and fewer headaches.
    \item Free and user friendly IDE. Thanks to Scala world for auto-completion,
      error highlight, navigation shortcut and many others advantages
    \item Extract information from your digital design and then generate files
      that contain information about some latency and addresses
    \item Bidirectional translation between any data type and bits. Useful to
      load a complex data structure from a CPU interface
    \item Check for you that there is no combinational loop / latch
    \item Check that there is no unintentional cross clock domain.
\end{itemize}

The listing \ref{lst:spinalhdl_example_and_gate} shows a AND gate written with
SpinalHDL with the corresponding generated VHDL code. We can see the
similarity between the two codes.

\begin{listing}[H] % {{{
    \centering

    \begin{minipage}[c]{0.45\textwidth}
    \begin{scalacode}
    import spinal.core._

    class AND extends Component
    {
        val io = new Bundle
        {
            val a = in Bool
            val b = in Bool
            val c = out Bool
        }

        io.c := io.a & io.b
    }
    \end{scalacode}
    \end{minipage}
    \hfill
    \begin{minipage}[c]{0.45\textwidth}
    \begin{vhdlcode}
    entity AND_1 is
        port(
            io_a : in std_logic;
            io_b : in std_logic;
            io_c : out std_logic
        );
    end AND_1;

    architecture arch of AND_1 is

    begin
      io_c <= (io_a and io_b);
    end arch;
    \end{vhdlcode}
    \end{minipage}
    \caption[SpinalHDL example and generated VHDL code]{Example of a AND gate written in SpinalHDL and the corresponding
             generated VHDL code}
    \label{lst:spinalhdl_example_and_gate}
\end{listing}

\section{Goal}
\label{sec:Goal}

The goal of the project is to produce an application which will analyse a
SpinalHDL program in order to produce a block diagram of the corresponding
hardware description. This application should offer the following activities :

\begin{itemize}
    \item Display the complete graph of the program
    \item Navigate through all the hierarchical levels of the program (Components
      inside another component)
    \item The edges should display some information about the signals :
    \begin{itemize}
        \item type
        \item input or output
        \item ...
    \end{itemize}
    \item A hierarchical view (like in filesystems or in Quartus)
    \item A filter functionality in order to view only some specifics components
\end{itemize}

The application is going to be developed, firstly, in a standalone version and
next could be develop into a plugin version for Eclipse or IntelliJ.

\section{Document overview}
\label{sec:Document overview}

TODO

%%% Local Variables:
%%% mode: latex
%%% TeX-master: "../report"
%%% End: