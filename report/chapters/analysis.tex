\chapter{Analysis}
\label{cha:Analysis}

In this chapter, we would discuss about the "What do we want ?" questions. It
means what does need the final product based on the goal of the project. The other
part is about the actual state of SpinalHDL. We are not presenting the whole
implementation or conception, but a small part is useful for this
project. Finally, we introduce the concept of the SpinalHDL's component and AST
(Abstract Syntax Tree).

\section{What do we want ?}
\label{sec:What do we want ?}

Based on the goal of the project at section \ref{sec:Goal}, we expose what the
end user of SpinalHDL wants to do with KlugHDL. The figure
\ref{fig:use_case_klughdl} shows the use case of KlugHDL.

\begin{figure}[H]
    \centering

    \begin{tikzpicture}

        \umlactor[x=0,y=3,name=user]{SpinalHDL user}
        \umlactor[x=0,y=0,name=spinal]{SpinalHDL}

        \begin{umlsystem}[x=3,y=0]{KlugHDL}
            \umlusecase[x=4,y=6,name=UCA,width=180]{Generate a component diagram}
            \umlusecase[x=4,y=4.9,name=UCB,width=180]{View a component diagram}
            \umlusecase[x=4,y=3.7,name=UCC,width=180]{Display, hide and toggle
              the hierarchical tree view of the program}
            \umlusecase[x=4,y=2.6,name=UCD,width=180]{Filter the hierarchical view}
            \umlusecase[x=4,y=1.5,name=UCE,width=180]{Enter a component}
            \umlusecase[x=4,y=0.4,name=UCF,width=180]{Exit of a component}
            \umlusecase[x=4,y=-0.8,name=UCG,width=180]{Display, hide and toggle the type of connections}
            \umlusecase[x=4,y=-1.9,name=UCH,width=180]{Generate the RTL}
        \end{umlsystem}

        \umlassoc{SpinalHDL user}{UCA}
        \umlassoc{SpinalHDL user}{UCB}
        \umlassoc{SpinalHDL user}{UCC}
        \umlassoc{SpinalHDL user}{UCD}
        \umlassoc{SpinalHDL user}{UCE}
        \umlassoc{SpinalHDL user}{UCF}
        \umlassoc{SpinalHDL user}{UCG}
        \umlassoc{SpinalHDL user}{UCH}
        \umlassoc{SpinalHDL}{UCA}
        \umlassoc{SpinalHDL}{UCB}
        \umlassoc{SpinalHDL}{UCH}

    \end{tikzpicture}

    \caption{Use case of KlugHDL}
    \label{fig:use_case_klughdl}
\end{figure}

An important point is the idea of port. If we want to design a HDL diagram, we
have to display the types of the connections between the components. We have at
least two alternatives :
\begin{itemize}
 \item display the types using the edges
 \item display the types using some ports attached to the component
\end{itemize}

We need to figure out which alternative is the best in order to display the
connection's type.

\section{SpinalHDL}
\label{sec:SpinalHDL}

As explained in chapter \ref{cha:Introduction}, SpinalHDL is a DSL for VHDL
programing. SpinalHDL runs with various internals constructions such as an AST or
a Component Tree. Those are the major elements needed for the diagram generation.

\subsection{Component}
\label{sub:Component}

A SpinalHDL's component is the base class of every SpinalHDL program, like in
the listing \ref{lst:spinalhdl_example_and_gate}. Like in Verilog and VHDL, we
can define components that could be used to build a design hierarchy. But unlike
them, we don’t need to bind them at instantiation \cite{github-spinalhdl}.
That's why there is just one component in SpinalHDL versus the architecture and entity
combination in VHDL.

The component is the main part of the SpinalHDL architecture for this project
because all the components are linked together with :
\begin{itemize}
    \item a parent-children relationship
    \item a brother relationship
\end{itemize}

A component could contain and communicate with other components, that's the
parent-children relationship, and a component that could communicate with the
component contained by its parent, is a brother's relationship. Those two
relations could be seen in listing \ref{lst:HierarchicComponent-solo}. The
component communicates with its children which are communicating together.

\begin{listing}
    \centering
    \begin{scalacode}
    class ParentChildrenBrother extends Component
    {
        val andGate = new AndGate
        val xorGate = new XorGate
        val andXorGate1 = new AndXorGate
        val orGate = new OrGate
        val andXorGate2 = new AndXorGate

        val io = new Bundle
        {
            val a = in Bool
            val b = in Bool
            val c = in Bool
            val d = out Bool
            val e = out Bool
        }

        andGate.io.a := io.a            // children communication example
        andGate.io.b := io.b
        xorGate.io.a := io.b
        xorGate.io.b := io.c

        orGate.io.a := andGate.io.c     // brother communication example
        orGate.io.b := xorGate.io.c     

        andXorGate1.io.a := io.a
        andXorGate1.io.b := io.b

        andXorGate2.io.a := io.a
        andXorGate2.io.b := io.b

        io.d := andGate.io.c & xorGate.io.c & orGate.io.c
        io.e := andXorGate1.io.c | andXorGate1.io.d | andXorGate2.io.c | andXorGate2.io.d
    }
    \end{scalacode}
    \caption[Type of connection in SpinalHDL]{There are two different types of
connections with SpinalHDL, from brother to brother or from a parent to
one of its children. The corresponding diagram to this code is shown in figure
\ref{fig:hierarchical-layout-simple}}
    \label{lst:HierarchicComponent-solo}
\end{listing}

A component may have other elements inside it too, like Area or Function. But
they are not relevant for the diagram generation.

\subsection{Abstract Syntax Tree}
\label{sub:Abstract Syntax Tree}

An abstract syntax tree is a tree where the nodes are marked by operator and the
leaves are marked by the operand of those operators. For example, the expression $5
* 2 + 3$ has its equivalent AST shown in figure \ref{fig:ast-example}.

\begin{figure}[H]
    \centering
    \fbox{
        \digraph[scale=0.5]{AstExample}{
            rankdir=BT;
            plus[label="+"];
            mul[label="*"];
            2 -> plus;
            3 -> plus;
            5 -> mul;
            plus -> mul;
        }
    }
    \caption[Example of an AST]{AST of the expression $5 * 2 + 3$.}
    \label{fig:ast-example}
  \end{figure}

The understanding of the AST is the main concept of this project. With it, we are
able to generate the whole model for further implementations.

%%% Local Variables:
%%% mode: latex
%%% TeX-master: "../report"
%%% End:
