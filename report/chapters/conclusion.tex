\chapter{Conclusion}
\label{cha:Conclusion}

This project tries to offer an additional tool for SpinalHDL in order to
increase his popularity. Domain specific languages are more and more used in
computer software nowadays and KlugHDL could be described as a tool for
development and documentation.

Writing VHDL code over and over again seems to be painful and SpinalHDL tries to
make the developer life easier, but VHDL owns a lot of tools to increase
productivity. KlugHDL is the first ``tool'' of the whole world of SpinalHDL and
launches a new era for SpinalHDL and its believers.

KlugHDL is able to parse and generate static diagrams with DOT and Graphviz for
any simple SpinalHDL component. Components with bus or complex basetype, like
$Uint(n bits)$ aren't supported for the moment. KlugHDL is also here to
demonstrate that it is possible to generate diagrams from the SpinalHDL AST.

We have seen that parsing the AST could be tricky and sometimes really strange.
We need additionnal SpinalHDL Components and much deeper understanding of the
SpinalHDL AST in order to parse and generate all the possible components which
could be written.

There is a lot of work to do on KlugHDL in order to have a complete tool for
diagrams generation. It is not only about the visualization but also about the
kind of SpinalHDL components to parse.

Some other features and work can be add to KlugHDL :
\begin{itemize}
\item Integrate the diagram visualization to an integrated software development
  like Eclipse or Intellij for a live preview of the current work.
\item Generate the diagram without generating the whole RTL for a component.
\end{itemize}

I would like to sincerely thanks the following people, whitout whom this project
would have been impossible to realize :

\begin{itemize}
\item Professor Dr. Mudry Pierre-André, for coordinating and supporting me
\item Charles Papon, author of SpinalHDL, for answered my questions about
  SpinalHDL and supporting me
\item Roland Julmy, my father, for having review this document and encouraging me.
\end{itemize}

Fribourg, the 17th January 2017

Sylvain Julmy

\chapter*{Sworn declaration}

`` I hereby solemnly declare that I have personally and independently prepared
this paper. All quotations in the text have been marked as such, and the paper
or considerable parts of it have not previously been subject to any examination
or assessment.''

\listoffigures
\listoflistings

%%% Local Variables:
%%% mode: latex
%%% TeX-master: "../report"
%%% End: