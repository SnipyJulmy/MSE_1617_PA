\chapter{Parsing the AST}
\label{chap:parsing-ast}

The abstract syntax tree of SpinalHDL is the core components of the whole KlugHDL
product. It is used to produce the intermediate representation in the generation
pipeline (figure \ref{fig:generation-pipeline}) and once the intermediate
representation is built, the rest is easily doable.

The concept of AST has already been explained in chapter \ref{sub:Abstract Syntax
  Tree}. Using this, we are going to see the parsing of the SpinalHDL AST in
order to generate an intermediate model. The parsing and the following
generation into the intermediate model is separate in multiple
phases :

\begin{itemize}
\item Diagram parsing
\item Components parsing
\item Ports parsing
\item Connections parsing
\end{itemize}

As illustrated in the intermediate model class diagram
\ref{fig:model-class-diagram}, the intermediate representation is composed as follows:
\begin{itemize}
\item A model possesses multiple diagrams (at least two : the ``toplevel'' component and the
  root component)
\item A diagrams possesses at least one component with zero or more ports
\item A diagrams possesses zero or more connections which are only brother to
  brother connections
\end{itemize}

A model could be represented, using Scala, with the class declaration in listing
\ref{lst:model-scala}. A model is a set of diagrams which are related to a
toplevel component.

\begin{listing}[H]
  \centering
  \begin{scalacode}
    case class Model(topLevel : Component) {

      var diagrams : Set[Diagram] = Set()
  }
\end{scalacode}
\caption[Model class declaration]{Declaration of the model with Scala, a model
  is basically a set of diagrams and is attached to a toplevel component, which
  is the only component of the AST which has no parent}
\label{lst:model-scala}
\end{listing}

\section{Diagram parsing}
\label{sec:diagrams-parsing}

Generating a diagram (see listing \ref{lst:diagram-scala}) is trivial because we
just have to retrieve all the components which are parents, the algorithm is
implemented in listing \ref{lst:generate-diagram}. 
\begin{listing}[H]
  \centering
  \begin{scalacode}
    class Diagram(val parent : Component) {

      var components : Map[Component, KlugHDLComponent] = Map()
      var connections : mutable.MultiMap[
          (KlugHDLComponent,Port),
          (KlugHDLComponent, Port)
        ]
      }
  \end{scalacode}
  \caption[Diagram class declaration]{Declaration of a the diagram class with
    Scala, a diagram is a set of components (here as a Map) and a set of
    connections (here as a multimap for the orientation)}
  \label{lst:diagram-scala}
\end{listing}

\begin{listing}[H]
  \centering
  \begin{scalacode}
  def generateDiagrams(component : Component) : Unit = {
    diagrams += new Diagram(component.parent)
    component.children.foreach(generateDiagrams)
  }
  \end{scalacode}
  \caption[Parsing the diagrams form the AST]{This function parse the AST and
    generate all the corresponding diagrams Object for a specific component}
  \label{lst:generate-diagram}
\end{listing}

Because we are using a set,
we don't have to check if the digram already exist. To make this possible we
have to redefine the \verb|equals| function like in listing \ref{lst:diagram-equals-scala}.

\begin{listing}[H]
  \centering
  \begin{scalacode}
  override def equals(o : scala.Any) : Boolean = o match {
    case d : Diagram => this.parent == d.parent
    case _ => super.equals(o)
  }
  \end{scalacode}
  \caption[Equals function implementation for the Diagram Class]{We have to
    override the equals function in order to use the diagrams in a set as expected}
  \label{lst:diagram-equals-scala}
\end{listing}

\section{Components parsing}
\label{sec:components-parsing}

The parsing and generation of the components is made for each diagrams one by
one. Here, the algorithm is quite simple too. The algorithm is implemented in
the listing \ref{lst:component-parse-scala} and the KlugHDLComponent is declared
in listing \ref{lst:component-scala}.

\begin{listing}[H]
  \centering
  \begin{scalacode}
  sealed trait KlugHDLComponent {

    val name : String
    val parent : Component
    val component : Component = this match {
      case KlugHDLComponentBasic(_, c, _) => c
      case KlugHDLComponentIO(_, _) => null
    }
  }

  final case class KlugHDLComponentBasic(name : String, override val component : Component, parent : Component) extends KlugHDLComponent

  final case class KlugHDLComponentIO(name : String, parent : Component) extends KlugHDLComponent
  \end{scalacode}
  \caption[Diagram class declaration]{Declaration of the Components class (here
    names KlugHDLComponents). There are two type of components : the basic ones
    and the ones who represent the external world input and output like in
    figure \ref{fig:and-xor-gate}}
  \label{lst:component-scala}
\end{listing}

\begin{listing}[H]
  \centering
  \begin{scalacode}
  private def generateComponent(diagram : Diagram) : Model = {
    diagram.foreachChildren(diagram.addComponents, topLevel)
    if (diagram.parent != null)
      diagram.addIoComponents(diagram.parent)
    this
  }

  def addComponents(component : Component) : Unit = {
    components += (component -> KlugHDLComponentBasic(component.definitionName, component, component.parent))
  }
  
  def addIoComponents(component : Component) : Unit = {
    if (component == null)
      components += (component -> KlugHDLComponentIO("NULL", component))
    else
      components += (component -> KlugHDLComponentIO(component.definitionName, component))
  }
  \end{scalacode}
  \caption[Implementation of the components parsing]{Components parsing and
    generation in Scala for one diagram}
  \label{lst:component-parse-scala}
\end{listing}

Note that we generate the interface for the external world connections in the
model. This two KlugHDLComponentIO aren't present in the SpinalHDL AST.

\section{Ports parsing}
\label{sec:ports-parsing}

The ports generation is, again, easy to make. In the SpinalHDL model, the
Components class already owns a \verb|getAllIo| function which return all the
inputs and outputs node for a components. So we just need to create a ports
object (listing \ref{lst:port-scala}) from these and add it to the correct
KlugHDLComponents. The algorithm is implemented in the listing
\ref{lst:port-parsing-scala}.

\begin{listing}[H]
  \centering
  \begin{scalacode}
  sealed trait Port {

    val name : String
    val hdlType : String
  }

  final case class InputPort(name : String, hdlType : String) extends Port

  final case class OutputPort(name : String, hdlType : String) extends Port
  \end{scalacode}
  \caption[Diagram class declaration]{Declaration if the Port class with Scala,
    there is two types of ports : input and output.}
  \label{lst:port-scala}
\end{listing}

\begin{listing}[H]
  \centering
  \begin{scalacode}
  def generatePort(diagram: Diagram): Unit = {
    def generatePort(entry: (Component, KlugHDLComponent)): Unit = {
      if (entry._1 != null) {
        entry._1.getAllIo.foreach { bt =>
          entry._2.addPort(Port(bt))
        }
      }
    }

    diagram.components.foreach(generatePort)
  }
  \end{scalacode}
  \caption[Implementation of the ports parsing]{Ports parsing and generation in
    Scala for one diagrams, the ports are generating component by component.}
  \label{lst:port-parsing-scala}
\end{listing}

\section{Connections parsing}
\label{sec:connections-parsing}

The connections parsing is the hardest implementation to do. The reason we
have multiples types of connections :
\begin{itemize}
\item input connections
\item output connections
\item input connections from parent
\item output connections from parent
\end{itemize}

In addition, to find the starting point of a connection, we have to parse
recursively all the nodes of the AST until we found a \verb|Basetype| object
which is the type for the inputs and outputs nodes of a SpinalHDL component.

\subsection{Input connections}
\label{sec:input-connections}

The input connections represent all the connections generated from the inputs of a
nodes. Firstable, we have to retrieve all the inputs for a specific AST node
then we need to generate a connection between two ports on two nodes. The
implementation can be seen in listing \ref{lst:parse-input-connection-scala}.

\begin{listing}[H]
  \centering
  \begin{scalacode}
  def parseInputConnection(component: Component): Unit = {
      
    def parseInputs(node: Node): List[BaseType] = {
      
      // iterate over all the inputs on each nodes of the AST until finding an
      // output basetype
      def inner(node: Node): List[BaseType] = node match {
        case bt: BaseType =>
          if (bt.isOutput) List(bt)
          else List(bt) ::: node.getInputs.map(inner).foldLeft(List(): List[BaseType])(_ ::: _)
        case null => List()
        case _ => node.getInputs.map(inner).foldLeft(List(): List[BaseType])(_ ::: _)
      }

      inner(node).filter {
        _ match {
          case bt: BaseType => bt.isOutput
          case _ => false
        }
      }
    }

    for {
      io <- component.getAllIo
      if io.isInput
      input <- parseInputs(io)
    } {
      diagram.addConnection(input.component, Port(input), io.component, Port(io))
    }
  }
  \end{scalacode}
  \caption[Parsing and generation of the inputs connections]{Implementation in
    Scala of the parsing and generation for all the inputs connections for a
    specific component}
  \label{lst:parse-input-connection-scala}
\end{listing}

\subsection{Output connections}
\label{sec:output-connections}

The output connections represent all the connections generated from the outputs
of a node. The problem is the same as for the input connections but in the other
way : we need to parse the consumer for a specific node like in listing.

\begin{listing}[H]
  \centering
  \begin{scalacode}
  def parseOutputConnection(component: Component): Unit = {
      
    def parseConsumers(node: Node): List[BaseType] = {
      
      // iterate over all the inputs on each nodes of the AST until finding an
      // input basetype
      def inner(node: Node): List[BaseType] = node match {
        case bt: BaseType =>
          if (bt.isInput) List(bt)
          else List(bt) ::: node.consumers.map(inner).foldLeft(List(): List[BaseType])(_ ::: _)
        case null => List()
        case _ => node.consumers.map(inner).foldLeft(List(): List[BaseType])(_ ::: _)
      }

      inner(node).filter {
        _ match {
          case bt: BaseType => bt.isInput
          case _ => false
        }
      }
    }

    for {
      io <- component.getAllIo
      if io.isInput
      consumer <- parseConsumers(io)
    } {
      diagram.addConnection(io.component, Port(io), consumer.component, Port(consumer))
    }
  }
  \end{scalacode}
  \caption[Parsing and generation of the outputs connections]{Implementation in
    Scala of the parsing and generation for all the inputs connections for a
    specific component}
  \label{lst:parse-output-connection-scala}
\end{listing}

\subsection{Parent connections}
\label{sec:parent-connections}

Finally we could parse and generate the connections with the parent. This part
is slightly easier than the two previous one. Because we just have to generate
the inputs and consumers and subtract them like in a set. The algorithm is
described in listing \ref{lst:parse-parent-connection}.

\begin{listing}[H]
  \centering
  \begin{scalacode}
  def parseParentConnection(component: Component): Unit = {

    def parseInputParentConnection(component: Component): Unit = {
      if (component.parent != null) {
        val con = for {
          io_p <- component.parent.getAllIo
          io <- component.getAllIo
          if io.getInputs.contains(io_p)
        } yield (io_p.component, Port(io_p), io.component, Port(io))
        con.foreach(diagram.addConnection)
      }
    }

    def parseOutputParentConnection(component: Component): Unit = {
      if (component.parent != null) {
        val con = for {
          io_p <- component.parent.getAllIo
          io <- getInputs(io_p)
          if io != null
        } yield (io.component, Port(io), io_p.component, Port(io_p))
        con.foreach(diagram.addConnection)
      }
    }

    def getInputs(bt: BaseType): List[BaseType] = {
      val comp = bt.component

      def inner(n: Node, acc: List[Node]): List[Node] = {
        if (n == null) acc
        else if (n.component != comp) {
          acc
        }
        else if (n.getInputsCount > 1) {
          n.getInputs.flatMap(n => inner(n, Nil)).toList
        }
        else {
          inner(n.getInputs.next(), n :: acc)
        }
      }

      inner(bt, Nil).map(_.getInputs.next().asInstanceOf[BaseType])
  }

    parseInputParentConnection(component)
    parseOutputParentConnection(component)
  }
\end{scalacode}
  \caption[Parsing and generation of the connections with the
  parent]{Implementation in Scala of the parsing and generation of all the
    connections (inputs and outputs ones) with the parent}
  \label{lst:parse-parent-connection}
\end{listing}

\section{Conclusion}
\label{sec:ast-parsing-conclusion}

The parsing of the AST is the hardest part to understand, there is a lot to know
about SpinalHDL itself and a large amount of specials cases to understand. By the
way, it's the most important part too, with the generation of the
intermediate model we could do all the rest.

%%% Local Variables:
%%% mode: latex
%%% TeX-master: "../report"
%%% End:
