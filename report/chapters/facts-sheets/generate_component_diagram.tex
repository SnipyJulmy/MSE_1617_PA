\begin{flushleft}
    \textsc{\huge Generate a component diagram}\\[0.5cm]

    \BlackLine
    \textsc{\Large 1 : Identification summary}\\[0.3cm]

        \textbf{\large Abstract} : The user wants to generate the component diagram from the code \\[0.1cm]

        \textbf{\large Authors} : Sylvain Julmy \\[0.3cm]			

        \textbf{\large Actors} : SpinalHDL user \\[0.1cm]	
    \begin{minipage}{0.40\textwidth}
        \begin{flushleft}	
            \textbf{\large Creation date} : 12.10.2016 \\[0.1cm]

            \textbf{\large Modification date} : 18.10.2016 \\[0.1cm]
        \end{flushleft}
    \end{minipage}
    \begin{minipage}{0.40\textwidth}
        \begin{flushleft}
            \textbf{\large Version} : 0.1 \\[0.1cm]

            \textbf{\large Responsible} : Sylvain Julmy \\[0.1cm]
        \end{flushleft}
    \end{minipage}
    \\[0.5cm]
    \BlackLine
    \textsc{\Large 2 : Sequence description}\\[0.3cm]

    \textbf{\large Precondition :} The user activates the compilation options

    \textbf{\large  Standard progress :}
    \begin{enumerate}[nosep]
        \item The user indicates to the compiler to launch KlugHDL
        \item The user compiles the program using the standard Scala compiler
        \item A Windows appears and shows the component diagram
    \end{enumerate}

    \textbf{\large  Alternative sequence :}\\
    A1 : Start at point 2 of standard progress
    \begin{enumerate}[nosep]
        \item There is a compilation error
        \item Nothing happens by KlugHDL
        \item Restart at point 1 of standard progress
    \end{enumerate}

    \textbf{\large Postcondition :} The component diagram is displayed

    \BlackLine
    \textsc{\Large 3 : HCI needing (Optional)}\\[0.3cm]

    A windows that shows the component diagram

    \BlackLine
    \textsc{\Large 4 : Complementary remarks}\\[0.3cm]

    KlugHDL is not really a standalone application, it's more like an option of
    SpinalHDL at compile time.

\end{flushleft}
