\begin{flushleft}
    \textsc{\huge View the component diagram}\\[0.5cm]

    \BlackLine
    \textsc{\Large 1 : Identification summary}\\[0.3cm]

        \textbf{\large Abstract} : The user wants to display and view the component diagram of its program \\[0.1cm]

        \textbf{\large Authors} : Sylvain Julmy \\[0.3cm]			

        \textbf{\large Actors} : SpinalHDL user \\[0.1cm]	
    \begin{minipage}{0.40\textwidth}
        \begin{flushleft}	
            \textbf{\large Creation date} : 12.10.2016 \\[0.1cm]

            \textbf{\large Modification date} : 21.10.2016 \\[0.1cm]
        \end{flushleft}
    \end{minipage}
    \begin{minipage}{0.40\textwidth}
        \begin{flushleft}
            \textbf{\large Version} : 0.1 \\[0.1cm]

            \textbf{\large Responsible} : Sylvain Julmy \\[0.1cm]
        \end{flushleft}
    \end{minipage}
    \\[0.5cm]
    \BlackLine
    \textsc{\Large 2 : Sequence description}\\[0.3cm]

    \textbf{\large Precondition :} The RTL has been generated without compilation error and the user has entered the correct option for the compilation.

    \textbf{\large  Standard progress :}
    \begin{enumerate}[nosep]
        \item The user compiles his program
        \item The diagram is displayed by the system
    \end{enumerate}

    \textbf{\large Postcondition :} The diagram is displayed.

    \BlackLine
    \textsc{\Large 3 : HCI needing (Optional)}\\[0.3cm]

    \begin{itemize}
        \item A diagram of the component of the user program
    \end{itemize}

    \BlackLine
    \textsc{\Large 4 : Complementary remarks}\\[0.3cm]

    The diagram could be generated but not necessarily displayed by the user.

\end{flushleft}
