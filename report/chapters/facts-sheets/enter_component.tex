\begin{flushleft}
    \textsc{\huge Enter a component}\\[0.5cm]

    \BlackLine
    \textsc{\Large 1 : Identification summary}\\[0.3cm]

        \textbf{\large Abstract} : The user wants to enter inside a component (zoom in) in order to show the sub-component \\[0.1cm]

        \textbf{\large Authors} : Sylvain Julmy \\[0.3cm]

        \textbf{\large Actors} : SpinalHDL user \\[0.1cm]

    \begin{minipage}{0.40\textwidth}
        \begin{flushleft}	
            \textbf{\large Creation date} : 12.10.2016 \\[0.1cm]

            \textbf{\large Modification date} : 18.10.2016 \\[0.1cm]
        \end{flushleft}
    \end{minipage}
    \begin{minipage}{0.40\textwidth}
        \begin{flushleft}
            \textbf{\large Version} : 0.1 \\[0.1cm]

            \textbf{\large Responsible} : Sylvain Julmy \\[0.1cm]
        \end{flushleft}
    \end{minipage}
    \\[0.5cm]
    \BlackLine
    \textsc{\Large 2 : Sequence description}\\[0.3cm]

    \textbf{\large Precondition :} The diagram is generated and opened.

    \textbf{\large  Standard progress :}
    \begin{enumerate}[nosep]
        \item The User double clicks on a component
        \item The system displays the inside of the component
    \end{enumerate}

    \textbf{\large Postcondition :} The double clicked component is maximized and is now the "root" component of the window.

    \BlackLine
    \textsc{\Large 3 : HCI needing (Optional)}\\[0.3cm]
    \begin{itemize}
        \item A visualization of the component and the sub-component
    \end{itemize}

    \BlackLine
    \textsc{\Large 4 : Complementary remarks}\\[0.3cm]

    Except for the leaf component which has no sub-component and the BlackBox, every component is zoomable.

\end{flushleft}
