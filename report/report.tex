% ==============================
% Authors of the LaTeX template:
%   - Sylvain Julmy
%   - Marc Demierre
% ==============================

\documentclass[a4paper,11pt]{report}

% General tools
\usepackage{etoolbox}

\usepackage[utf8]{inputenc}

% Fonts
\usepackage[T1]{fontenc}

% LaTeX modern fonts
% \usepackage{lmodern}

% Sans serif
%\usepackage{tgheros}

% Serif
%\usepackage[bitstream-charter]{mathdesign}

% Monospace
\usepackage{sourcecodepro}

% Bibtex and biblatex for references
\usepackage[autostyle]{csquotes}
\usepackage[backend=biber,
    style=chem-angew,
    sortlocale=en_GB,
    natbib=true,
    url=false, 
    doi=true,
    eprint=false
]{biblatex}
\addbibresource{biblatex-res.bib}

% Glossary
\usepackage[xindy,toc]{glossaries}

% Language
\usepackage[english]{babel}
\usepackage{blindtext}

% Page style
\usepackage{fullpage} % page margins to 1.5cm
\usepackage{fancyhdr} % headers and footers

% Colors & graphics
\usepackage[table]{xcolor}    % colors
\usepackage[pdftex]{graphicx} % graphics importing

% Misc
\usepackage{titlesec} % section titles formatting
\usepackage{minted}   % code highlighting
\usepackage{lscape}   % landscape
\usepackage{tikz}     % charts in LaTeX
\usepackage{amsmath}  % better math
\usepackage{float}    % floats
\usepackage[small,justification=centering]{caption}
\usepackage{microtype} % typographic improvements

% Paragraphs
\usepackage{parskip}
\usepackage[defaultlines=3,all]{nowidow}

% Chapter titles
% Remove space before title
\titlespacing{\chapter}{0pt}{*-4}{*3}
% Remove "Chapter N" and use a sans-serif font
\titleformat{\chapter}[hang]{\bf\huge}{\thechapter}{1pc}{}
% Change chapter page style
\patchcmd{\chapter}{plain}{fancy}{}{}

% Tables
\usepackage{multirow}

% Cross-references
\usepackage{hyperref}

% Metadata
% --------
% ==============================
% Authors of the LaTeX template:
%   - Sylvain Julmy
%   - Marc Demierre
% ==============================

% Metadata for this report
% ------------------------
\newcommand{\School}{University of Applied Sciences Western Switzerland}
\newcommand{\Faculty}{MSE - Software Engineering}
\newcommand{\Place}{Fribourg}

% Course
\newcommand{\Course}{Deepening project}
\newcommand{\Title}{KlugHDL : a VHDL language generator}

% Supervisors (professors)
\newcommand{\Supervisors}{Mudry Pierre-André}

% Students
\newcommand{\Authors}{Julmy Sylvain}



% Header and footer
% -----------------
\pagestyle{fancy}
\lhead[]{\Course}
\chead[]{}
\rhead[]{\Place, \today}

\setlength{\headheight}{14pt}
\setlength{\headsep}{14pt}

\newcommand{\HRule}{\rule{\linewidth}{0.5mm}}

% Code styles for highlighting
% ----------------------------

% How to use (replace 'java' with language name):
% - code blocks:
%     \begin{javacode}
%     CODE
%     \end{javacode}
% - files:
%     full: \javafile{PATH}
%     extract: \javafile[startline=x, endline=y]{PATH}
% TODO: inline?

% Java
\newminted{java}{frame=single, framesep=6pt, breaklines=true, fontsize=\scriptsize}
\newmintedfile{java}{frame=single, framesep=6pt, breaklines=true, fontsize=\scriptsize}

% Scala
\newminted{scala}{frame=single, framesep=6pt, breaklines=true, fontsize=\scriptsize}
\newmintedfile{scala}{frame=single, framesep=6pt, breaklines=true, fontsize=\scriptsize}

% Python
\newminted{python}{frame=single, framesep=6pt, breaklines=true, fontsize=\scriptsize}
\newmintedfile{python}{frame=single, framesep=6pt, breaklines=true, fontsize=\scriptsize}

% Plain text
\newminted{text}{frame=single, framesep=6pt, breaklines=true, breakanywhere, fontsize=\scriptsize}
\newmintedfile{text}{frame=single, framesep=6pt, breaklines=true, breakanywhere, fontsize=\scriptsize}

% VHDL
\newminted{vhdl}{frame=single,framesep=6pt, breaklines=true, fontsize=\scriptsize}
\newmintedfile{vhdl}{frame=single, framesep=6pt, breaklines=true, fontsize=\scriptsize}

% Document
% --------
\begin{document}

\begin{titlepage}
    \begin{center}

        % only works if a paragraph has started.
        \includegraphics[width=0.8\textwidth]{img/mse_logo}~\\[1.5cm]
        \textsc{\Large \School}\\[0.25cm]
        \textsc{\Large \Faculty}\\[1.5cm]
        \textsc{\LARGE \Course}\\[0.5cm]

        % Title
        \HRule \\[0.4cm]
        { \huge \bfseries \Title\\[0.4cm] }
        \HRule \\[1.5cm]

        % Author and supervisor
        \begin{minipage}[t]{0.4\textwidth}
            \begin{flushleft} \Large
                \emph{Author:}\\ \Authors
            \end{flushleft}
        \end{minipage}
        \begin{minipage}[t]{0.4\textwidth}
            \begin{flushright} \Large
                \emph{Supervisor:}\\\Supervisors
            \end{flushright}
        \end{minipage}~\\[1.5cm]

        \begin{center}
            \includegraphics[scale=0.7]{img/logo_hes-so}
        \end{center}

        \vfill

        % Bottom of the page
        {\large \Place, \today}

    \end{center}
\end{titlepage}

% Uncomment this for a table of contents

\tableofcontents
\newpage

\chapter{Introduction} %{{{
\label{cha:Introduction}

VHDL is an electronic design and hardware design language used to describe digital and signals systems such as field-programmable gate arrays and integrated circuits\cite{wiki-vhdl}. It's an old langague (VHDL was created in the 80's) which is complicated, verbose and confusing (the double meaning of the "sequential" construction).

The SpinalHDL language has been created to offer a more plaisant way to code in VHDL and this project is a contribution for SpinalHDL has an additionnal tools which gravited around the language itself.

\section{Context} %{{{
\label{sec:Context}

SpinalHDL is a programming language to describe digital hardware and generate the corresponding in VHDL (or Verilog). SpinalHDL is written in Scala as a DSL (Domain Specific Language) and has multiple advantages\cite{github-spinalhdl} :
\begin{itemize}
    \item No restriction to the genericity of your hardware description by using Scala constructs
    \item No more endless wiring. Create and connect complex buses like AXI in one line.
    \item Evolving capabilities. Create your own buses definition and abstraction layer.
    \item Reduce code size by a high factor, especially for wiring. Allowing you to have a better visibility, more productivity and fewer headaches.
    \item Free and user friendly IDE. Thanks to scala world for auto-completion, error highlight, navigation shortcut and many others
    \item Extract information from your digital design and then generate files that contain information about some latency and addresses
    \item Bidirectional translation between any data type and bits. Useful to load a complex data structure from a CPU interface.
    \item Check for you that there is no combinational loop / latch
    \item Check that there is no unintentional cross clock domain 
\end{itemize}

The code \ref{lst:spinalhdl_example_and_gate} shows a AND gate written with SpinalHDL with the corresponding generated VHDL code, we could see the similarity between the two code.

\begin{listing} % {{{
    \centering
    
    \begin{minipage}[c]{0.45\textwidth}
    \begin{scalacode}
    import spinal.core._

    class AND extends Component
    {
        val io = new Bundle
        {
            val a = in Bool
            val b = in Bool
            val c = out Bool
        }

        io.c := io.a & io.b
    }    
    \end{scalacode}
    \end{minipage}
    \hfill
    \begin{minipage}[c]{0.45\textwidth}
    \begin{vhdlcode}
    entity AND_1 is
        port( 
            io_a : in std_logic;
            io_b : in std_logic;
            io_c : out std_logic 
        );
    end AND_1;

    architecture arch of AND_1 is

    begin
      io_c <= (io_a and io_b);
    end arch;
    \end{vhdlcode}
    \end{minipage}
    \caption{Example of a AND gate written in SpinalHDL and the corresponding generated VHDL code}
    \label{lst:spinalhdl_example_and_gate}
\end{listing} %}}}

%}}} section Context

\section{Goal} %{{{
\label{sec:Goal}

The goal of the project is to produce an application which is analysing a spinalhdl program in order to produce a block diagramm of the corresponding hardware description. This application should offer the following activities :

\begin{itemize}
    \item Display the complete graph of the programm
    \item Navigate through all the hierarchical level of the programm (Component inside another component)
    \item The edges should display some information about the signals :
    \begin{itemize}
        \item type
        \item input or output
        \item ...
    \end{itemize}
    \item A bidirectional navigability between the application and the source code
\end{itemize}

The application is going to be develop, firstable, in a standalone version and next could be develop into a plugin version for Eclipse or IntelliJ.

%}}} section Goal

\section{Document overview} %{{{
\label{sec:Document overview}

TODO

%}}} section Document overview

%}}} chapter Introduction 

\chapter{SpinalHDL} %{{{
\label{cha:SpinalHDL}

TODO

\section{AST} %{{{
\label{sec:AST}

TODO


%}}} section AST

\section{Internal representation} %{{{
\label{sec:Internal representation}

TODO


%}}} section Internal representation

\section{A first overview} %{{{
\label{sec:A first overview}

TODO


%}}} section A first overview

%}}} chapter SpinalHDL

\chapter{Viewing library} %{{{
\label{cha:Viewing library}

TODO


\section{Graphstream} %{{{
\label{sec:Graphstream}

TODO


%}}} section Graphstream

\section{D3.js and C3.js} %{{{
\label{sec:D3.js and C3.js}

TODO


%}}} section D3.js and C3.js

\section{Vis.js} %{{{
\label{sec:Vis.js}

TODO


%}}} section Vis.js

\section{JGraphT} %{{{
\label{sec:JGraphT}

TODO


%}}} section JGraphT

\section{GraphViz} %{{{
\label{sec:GraphViz}

TODO


%}}} section GraphViz

\section{Prefuse} %{{{
\label{sec:Prefuse}

TODO


%}}} section Prefuse

\section{Processing} %{{{
\label{sec:Processing}

TODO


%}}} section Processing

%}}} chapter Viewing library 

\chapter{Specification} %{{{
\label{cha:Specification}

TODO


\section{Goals} %{{{
\label{sec:Goals}

TODO


%}}} section Goals

\section{Use case} %{{{
\label{sec:Use case}

TODO


%}}} section Use case

\section{Facts sheets} %{{{
\label{sec:Facts sheets}

TODO


%}}} section Facts sheets


%}}} chapter Specification 

\chapter{Implementation} %{{{
\label{cha:Implementation}

TODO


%}}} chapter Implementation 

\chapter{Tests} %{{{
\label{cha:Tests}

TODO


%}}} chapter Tests 

\chapter{Conclusion} %{{{
\label{cha:Conclusion}

TODO


%}}} chapter Conclusion 

\printbibliography

\end{document}

